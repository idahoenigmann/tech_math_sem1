\documentclass[twocolumn]{article}
\usepackage[utf8]{inputenc}
\usepackage[german]{babel}

\usepackage{amsthm}
\usepackage{amsmath}
\usepackage{amsfonts}
\usepackage{mathtools}
\usepackage{centernot}

\newtheorem{theorem}{Satz}[section]
\newtheorem{corollary}{Corollary}[theorem]
\newtheorem{lemma}[theorem]{Lemma}
\newtheorem{definition}{Definition}[section]
\newtheorem*{remark}{Bemerkung}
\newtheorem*{schreibweise}{Schreibweise}

%opening
\title{Zusammenfassung Heft 3 ANA}
\author{Ida Hönigmann}

\newcommand*{\logeq}{\Leftrightarrow}

\begin{document}
	
\maketitle

\section{Metrische Räume}

\begin{definition}
	$<M,d>$ ... metrischer Raum, $\epsilon \in \mathbb{R}$, $\epsilon > 0$, $x \in M$
	
	$U_\epsilon(x)\coloneqq \{y \in M : d(x,y) < \epsilon\}$ heißt die offene $\epsilon$-Kugel um $x$.

	$U_\epsilon(x)\coloneqq \{y \in M : d(x,y) \leq \epsilon\}$ heißt die abgeschlossene $\epsilon$-Kugel um $x$.

\end{definition}

\begin{remark}
	$(x_n)_{n \in \mathbb{N}}$ ... Folge in $M$, $x \in M$
	
	$\lim\limits_{n\rightarrow \infty}x_n = x \logeq \forall \epsilon > 0 \exists N \in \mathbb{N} \{x_n:n\geq N\} \subseteq U_\epsilon(x)$
\end{remark}

\begin{lemma}
	$<M,d>$... metrischer Raum
	
	\begin{itemize}
		\item $n \in \mathbb{N}$, $O_1,...,O_n \subseteq M$ ... offen $ \implies \bigcap_{j=1}^{n}O_j$ ... offen
		\item $O_i$, $i \in I$ ... offene Teilmengen von $M \implies \bigcup_{i \in I}O_i$ ... offen
	\end{itemize}
\end{lemma}

\end{document}
